%%%%%%%%%%%%%%%%%%%%%%%%%%%%%%%%%%%%%%%%%
% Medium Length Graduate Curriculum Vitae
% LaTeX Template
% Version 1.1 (9/12/12)
%
% This template has been downloaded from:
% http://www.LaTeXTemplates.com
%
% Original author:
% Rensselaer Polytechnic Institute (http://www.rpi.edu/dept/arc/training/latex/resumes/)
%
% Important note:
% This template requires the res.cls file to be in the same directory as the
% .tex file. The res.cls file provides the resume style used for structuring the
% document.
%
%%%%%%%%%%%%%%%%%%%%%%%%%%%%%%%%%%%%%%%%%

%----------------------------------------------------------------------------------------
%	PACKAGES AND OTHER DOCUMENT CONFIGURATIONS
%----------------------------------------------------------------------------------------

\documentclass[]{res} % Use the res.cls style, the font size can be changed to 11pt or 12pt here
% \documentclass[a4paper]{article}

\usepackage{helvet} % Default font is the helvetica postscript font
% \usepackage{newcent} % To change the default font to the new century schoolbook postscript font uncomment this line and comment the one above
% \usepackage{baskervald}

\usepackage[brazil]{babel}
\usepackage[utf8]{inputenc}
% \usepackage[tmargin=2.2cm, bmargin=2.2cm, lmargin=2.2cm, rmargin=2.2cm]{geometry}
\usepackage[pdftex]{hyperref}
\hypersetup{
    colorlinks,
    citecolor=black,
    filecolor=black,
    linkcolor=black,
    urlcolor=black
}

\usepackage{calc}

\usepackage{enumitem}

% A4 tem 8.267in x 11.692in

% resumewidth default é 6.5

% nt = nr - vr + vt

\def \mymargin{0.5in}
\def \mymarginright{0.85in}
\def \mysectionwidth{70pt}

\newsectionwidth{\mysectionwidth}

\def \newresumewidth{8.267in - \mymargin - \mymarginright}
\def \newtextwidth{\newresumewidth - \mysectionwidth}

\usepackage[margin=\mymargin, right=0in, a4paper]{geometry}

\setlength{\textwidth}{\newtextwidth}
% \setlength{\textwidth}{5.1in}
\setlength{\resumewidth}{\newresumewidth}

\newcommand{\sbt}{\,\begin{picture}(-1,1)(-2,-3)\circle*{2}\end{picture}\ }

\def \divspace{6pt}
\def \myitemback{0.55cm}
\def \myitemsep{0pt}

% \def \mypositionface{\sc}
\def \mypositionface{\sl}
% \def \mypositionface{\it}
% \def \mypositionface{\uppercase}
% \def \mypositionface{\lfseries}
% \def \mypositionface{\mdseries}

\def \myprojectface{\it}

\def \myorgface{\sc}
% \def \myorgface{\sl}
% \def \myorgface{\it}
% \def \myorgface{\uppercase}
% \def \myorgface{\lfseries}
% \def \myorgface{\mdseries}

\begin{document}

%----------------------------------------------------------------------------------------
%	NAME AND ADDRESS SECTION
%----------------------------------------------------------------------------------------

\moveleft.5\hoffset\centerline{\huge\uppercase{Mikail Campos Freitas}} % Your name at the top

\moveleft.5\hoffset\centerline{\rule{2.5in}{.2pt}}\smallskip % Your address

\moveleft.5\hoffset\centerline{24 anos - End. Alameda Itupiranga, 309, Saúde} % Your address
\moveleft.5\hoffset\centerline{São Paulo, SP - 04294-090}
\moveleft.5\hoffset\centerline{Tel. Res. (11) 2352-2157 / Cel. (11) 97288-9984}
\moveleft.5\hoffset\centerline{mik@il-freitas.com}

%----------------------------------------------------------------------------------------

\begin{resume}

%----------------------------------------------------------------------------------------
%	OBJECTIVE SECTION
%----------------------------------------------------------------------------------------
 
\section{OBJETIVOS \hspace{\divspace} }

Aprender novas tecnologias, expandir conhecimentos e habilidades. Encontrar e resolver problemas difíceis e desafios. Entrar em contato com profissionais de qualidade alta para que possa haver uma troca de aprendizados.


%----------------------------------------------------------------------------------------
%	EDUCATION SECTION
%----------------------------------------------------------------------------------------

\section{FORMAÇÃO \hspace{\divspace} }

{\myprojectface Bacharelado em Ciência da Computação} \hfill 2008 — 2012 \\
{\myorgface Instituto de Matemática e Estatística} da {\myorgface Universidade de São Paulo} \\
Trabalho de Conclusão de Curso: Resolvendo o problema PSAT com auxílio da ferramenta de software livre MiniSat
\begin{itemize}[itemsep=\myitemsep,leftmargin=\myitemback]
\item[\sbt] integração em nível de código de ferramentas de software livre para a solução de uma variante probabilística do problema clássico SAT
\end{itemize}


%----------------------------------------------------------------------------------------
%   EDUCATION SECTION
%----------------------------------------------------------------------------------------

% gambiarra pra fazer um título com duas linhas recuado alinhado na direita
\section{PROJETOS DE \hspace{\divspace} }
\section{FACULDADE \hspace{\divspace} }
\vskip-\baselineskip  % mágica da gambi que faz o texto ser renderizado um linha pra cima

{\myprojectface Desenvolvimento de jogo utilizando ActionScript 3 e Realidade Aumentada} \hfill 2010 (6 meses) \\
{\mypositionface Programador Principal} \\
Jogo de estilo puzzle utilizando a linguagem ActionScript 3 juntamente com as bibliotecas: JigLib para simulação física; Papervision3D para a renderização e efeitos em cenários e objetos 3D; FLARManager para definição, detecção e utilização de marcadores de realidade aumentada. Desenvolvido em Windows.

{\myprojectface Desenvolvimento de aplicação interativa 3D simples utilizando C++ e OpenGL} \hfill 2012 (5 meses) \\
{\mypositionface Programador Solo} \\
Simulação interativa de física simples com o objetivo de modelar e possibilitar a navegação pelos prédios do Instituto de Matemática e Estatística da USP (ambiente externo) utilizando apenas OpenGL com a linguagem C++. Desenvolvido em Mac OS X.

% {\sl Business Applications Programmer} \hfill Fall 1990 \\
% Allied-Signal Bendix Friction Materials Division, Financial Planning Department, Latham, NY 

% \begin{itemize}[itemsep=\myitemsep,leftmargin=\myitemback]
% \item[\sbt] Developed four ``user friendly" forecasting systems each of which produces 18 to 139 individual reports. 
% \item[\sbt] Developed or improved almost all IFPS programs used for financial reports. 
% \end{itemize}

% {\sl Research Programmer} \hfill Summer 1990 \\
% Psychology Department, Rensselaer Polytechnic Institute 
% \begin{itemize}[itemsep=\myitemsep,leftmargin=\myitemback]
% \item[\sbt] Performed computer aided statistical analysis of data. 
% \end{itemize} 

% {\sl Assistant Manager} \hfill Summers 1988-89 \\
% Thunder Restaurant, Canton, CT
% \begin{itemize}[itemsep=\myitemsep,leftmargin=\myitemback]
% \item[\sbt] Recognized need for, developed, and wrote employee training manual. Performed various duties including cooking, employee training, ordering, and inventory control
% \end{itemize}

\section{PROJETOS \hspace{\divspace} }
\section{PESSOAIS \hspace{\divspace} }
\vskip-\baselineskip  % mágica da gambi que faz o texto ser renderizado um linha pra cima

{\myprojectface Jogo multiplataforma utilizando C++} \hfill 2012 — presente \\
{\mypositionface Programador Solo} \\
Jogo 2D com engine própria para física e efeitos de luz e sombra. Projeto multiplataforma usando a biblioteca multimídia SFML (C++), OpenGL e GLSL. Atualmente possui versão para Mac OS X e Windows 7.


%----------------------------------------------------------------------------------------
%	PROFESSIONAL EXPERIENCE SECTION
%----------------------------------------------------------------------------------------
 
\section{EXPERIÊNCIA \hspace{\divspace} }

{\mypositionface Desenvolvedor Pleno} \hfill 2013 — 2014 \\
{\myorgface VTX Brasil}

\begin{itemize}[itemsep=\myitemsep,leftmargin=\myitemback]
\item[\sbt] Participação na concepção e direção do desenvolvimento de produtos.
\item[\sbt] Desenvolvimento de {\it frontend} e {\it backend} em aplicativo iOS de mensagens instantâne-\\as e vídeo-chamadas.\hfill(6 meses)
\item[\sbt] Desenvolvimento de {\it frontend} e {\it backend} em aplicativo Android de multivisualização\\ de vídeos.\hfill(3 meses)
\item[\sbt] Desenvolvimento de {\it frontend} e {\it backend} em aplicativo iOS de engajamento em vare-\\jo através de tecnologia de ponta.\hfill(10 meses)
\end{itemize}

{\mypositionface Líder Técnico de P\&D} \hfill 2014 — presente \\
{\myorgface VTX Brasil}

\begin{itemize}[itemsep=\myitemsep,leftmargin=\myitemback]
\item[\sbt] Planejamento e desenvolvimento de sistemas inteligentes de aprendizado computaci-\\onal com ênfase em processamento de imagens.\hfill(2 meses)
\end{itemize}

% App para iOS de IM e chamadas de vídeo  —  2013 (6 meses) \\
% Aplicativo de mensagens instantâneas e de chamadas de vídeo utilizando a biblioteca Liblinphone para gerenciamento de comunicação entre devices por meio do protocolo SIP.

% {\sl Business Applications Programmer} \hfill Fall 1990 \\
% Allied-Signal Bendix Friction Materials Division, Financial Planning Department, Latham, NY 

% \begin{itemize}[itemsep=\myitemsep,leftmargin=\myitemback]
% \item[\sbt] Developed four ``user friendly" forecasting systems each of which produces 18 to 139 individual reports. 
% \item[\sbt] Developed or improved almost all IFPS programs used for financial reports. 
% \end{itemize}
 
% {\sl Research Programmer} \hfill Summer 1990 \\
% Psychology Department, Rensselaer Polytechnic Institute 
% \begin{itemize}[itemsep=\myitemsep,leftmargin=\myitemback]
% \item[\sbt] Performed computer aided statistical analysis of data. 
% \end{itemize} 

% {\sl Assistant Manager} \hfill Summers 1988-89 \\
% Thunder Restaurant, Canton, CT
% \begin{itemize}[itemsep=\myitemsep,leftmargin=\myitemback]
% \item[\sbt] Recognized need for, developed, and wrote employee training manual. Performed various duties including cooking, employee training, ordering, and inventory control. 
% \end{itemize} 


%----------------------------------------------------------------------------------------
%   COMPUTER SKILLS SECTION
%----------------------------------------------------------------------------------------

\section{HABILIDADES \hspace{\divspace} }

% {\sl Languages \& Software:} 
% COBOL, IFPS, Focus, Megacalc, Pascal, Modula2, C, APL, SNOBOL, FORTRAN, LISP, SPIRES, BASIC, VSPC Autotab, IBM 370 Assembler, Lotus 1-2-3. \\
% {\sl Operating Systems:} MTS, TSO, Unix. 

- {\sl Linguagens}: \\
\setlength\tabcolsep{2pt}
\begin{tabular}{l l}
\hspace*{1em}{\it Fluente}:& Objective-C, C, regex \\
\hspace*{1em}{\it Muito Familiar}:& Python, C++, Java, Scheme, Erlang, ActionScript 3, \LaTeX, GLSL \\
\hspace*{1em}{\it Pouco Familiar}:& CUDA, Assembly, Prolog, Swift, Smalltalk, Pascal, Bash \\
\end{tabular}

- {\sl Frameworks, bibliotecas e tecnologias:} \\
\hspace*{1em}iBeacon, SFML, Allegro, NumPy, JigLib, Papervision3D, FLARManager, AllJoyn, Vuforia, OpenCV, \\
\hspace*{1em}Liblinphone \\

- {\sl Softwares e ferramentas:} \\
\hspace*{1em} Git (avançado), gnuplot, Adobe Photoshop, Xcode, Eclipse, ADT, \\
- Conhecimento intermediário de OpenGL e {\it shaders} \\
- Desenvolvimento em Ubuntu e Debian (7 anos), e Mac OSX (5 anos) \\
- Conhecimento dos IDEs Xcode, Eclipse, ADT e FlashDevelop \\
- Conhecimento intermediário de criptografia \\
- Edição básica de imagem no Adobe Photoshop \\
- Inglês fluente, japonês básico, espanhol básico \\



%----------------------------------------------------------------------------------------
%	EXTRA-CURRICULAR ACTIVITIES SECTION
%----------------------------------------------------------------------------------------

% gambiarra pra fazer um título com duas linhas recuado alinhado na direita
\section{PRÊMIOS E \hspace{\divspace} }
\section{CONQUISTAS \hspace{\divspace} }
\vskip-\baselineskip  % mágica da gambi que faz o texto ser renderizado um linha pra cima

% Elected {\it House Manager}, Rho Phi Sorority \\
% Elected {\it Sports Chairman} \\
% Attended Krannet Leadership Conference \\
% Headed delegation to Rho Phi Congress \\
% Junior varsity basketball team \\
% Participant, seven intramural athletic teams 

- Medalha de Bronze na {\it Olimpíada Brasileira de Informática} (2007) \\
- Medalha de Prata na {\it Olimpíada Brasileira de Astronomia} (2006) \\
- Medalha de Prata na {\it Olimpíada Brasileira de Astronomia} (2007) \\
- Menção Honrosa na {\it Olimpíada Brasileira de Física} (2007) \\
- Chamado até a última fase do recrutamento da Microsoft no IME - USP (2012)



%----------------------------------------------------------------------------------------

\end{resume}
\end{document}