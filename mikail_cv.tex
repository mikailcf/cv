%%%%%%%%%%%%%%%%%%%%%%%%%%%%%%%%%%%%%%%%%
% Medium Length Graduate Curriculum Vitae
% LaTeX Template
% Version 1.1 (9/12/12)
%
% This template has been downloaded from:
% http://www.LaTeXTemplates.com
%
% Original author:
% Rensselaer Polytechnic Institute (http://www.rpi.edu/dept/arc/training/latex/resumes/)
%
% Important note:
% This template requires the res.cls file to be in the same directory as the
% .tex file. The res.cls file provides the resume style used for structuring the
% document.
%
%%%%%%%%%%%%%%%%%%%%%%%%%%%%%%%%%%%%%%%%%

%----------------------------------------------------------------------------------------
%	PACKAGES AND OTHER DOCUMENT CONFIGURATIONS
%----------------------------------------------------------------------------------------

\documentclass[]{res} % Use the res.cls style, the font size can be changed to 11pt or 12pt here
% \documentclass[a4paper]{article}

\usepackage{helvet} % Default font is the helvetica postscript font
% \usepackage{newcent} % To change the default font to the new century schoolbook postscript font uncomment this line and comment the one above
% \usepackage{baskervald}

\usepackage[brazil]{babel}
\usepackage[utf8]{inputenc}
% \usepackage[tmargin=2.2cm, bmargin=2.2cm, lmargin=2.2cm, rmargin=2.2cm]{geometry}
\usepackage{hyperref}
\hypersetup{
    colorlinks,
    citecolor=black,
    filecolor=black,
    linkcolor=blue,
    urlcolor=blue
}

\usepackage{calc}

\usepackage{enumitem}

% A4 tem 8.267in x 11.692in

% resumewidth default é 6.5

% nt = nr - vr + vt


%----------------------------------------------------------------------------------------
%   DEFINIÇÕES
%----------------------------------------------------------------------------------------

\def \mymargin{0.4in}
\def \mymarginright{0.85in}
\def \mysectionwidth{70pt}

\newsectionwidth{\mysectionwidth}

\def \newresumewidth{8.267in - \mymargin - \mymarginright}
\def \newtextwidth{\newresumewidth - \mysectionwidth}

\usepackage[margin=\mymargin, top=0.7in, a4paper]{geometry}

\setlength{\textwidth}{\newtextwidth}
% \setlength{\textwidth}{5.1in}
\setlength{\resumewidth}{\newresumewidth}

\newcommand{\sbt}{\,\begin{picture}(-1,1)(-2,-3)\circle*{2}\end{picture}\ }

\def \divspace{6pt}
\def \myitemback{0.55cm}
\def \myitemsep{0pt}

% \def \mypositionface{\sc}
\def \mypositionface{\sl}
% \def \mypositionface{\it}
% \def \mypositionface{\uppercase}
% \def \mypositionface{\lfseries}
% \def \mypositionface{\mdseries}

\def \myprojectface{\it}

\def \myorgface{\sc}
% \def \myorgface{\sl}
% \def \myorgface{\it}
% \def \myorgface{\uppercase}
% \def \myorgface{\lfseries}
% \def \myorgface{\mdseries}

% parâmetros de foco do CV

\newif\ifgames
\gamestrue
%\gamesfalse

%---------------------------------------------------------------------------------------

\begin{document}

%----------------------------------------------------------------------------------------
%	NAME AND ADDRESS SECTION
%----------------------------------------------------------------------------------------

\moveleft.5\hoffset\centerline{\huge\uppercase{Mikail Campos Freitas}} % Your name at the top

\moveleft.5\hoffset\centerline{\rule{2.5in}{.2pt}}\smallskip % Your address

\moveleft.5\hoffset\centerline{Addr. Alameda Itupiranga, 309, Saúde} % Your address
\moveleft.5\hoffset\centerline{Brazil, São Paulo, SP - 04294-090}
\moveleft.5\hoffset\centerline{Home. +55 (11) 2352-2157 / Mobile. +55 (11) 97288-9984}

\ifgames
    \moveleft.5\hoffset\centerline{contact@mikail.io / Website. \href{http://mikail.io}{http://mikail.io}}
\else
    \moveleft.5\hoffset\centerline{contact@mikail.io}
\fi


\vskip-3.5ex

%-----------------------------------------------------------------------------------------------------------------------------------------------------------------------------

\begin{resume}

\section{OBJECTIVES \hspace{\divspace} }

To learn, obtain new skills and hone current ones. To earn experience through pursuit and overcoming of challenges as well as throughout the contact with professionals from the same and different areas, learning and teaching in the process.

%-----------------------------------------------------------------------------------------------------------------------------------------------------------------------------

\section{EDUCATION \hspace{\divspace} }

{\myprojectface Bachelor of Computer Science} \hfill 2008 — 2012 \\
{\myorgface University of São Paulo}'s {\myorgface Institute of Mathematics and Statistics} (IME - USP)\\
Course Completion Assignment: {\sl Solving the PSAT problem with assistance of the free software MiniSat} \href{http://www.ime.usp.br/~cef/mac499-12/monografias/rec/mikail/index.html}{[link (portuguese only)]}
\begin{itemize}[itemsep=\myitemsep,leftmargin=\myitemback]
\item[\sbt] Source code integration of free software tools to optimize one solution to a probabilistic variant of the SAT problem.
\end{itemize}

%-----------------------------------------------------------------------------------------------------------------------------------------------------------------------------

\section{COLLEGE \hspace{\divspace} }
\section{PROJECTS \hspace{\divspace} }
\vskip-2.5ex  % mágica da gambi que faz o texto ser renderizado uma linha pra cima

\ifgames
    {\myprojectface Simple 2D game with C} \hfill 2010 (4 months) \\
    {\mypositionface Lead Programmer} \\
    2D game with simple physics where two players compete with each other for points for saving survivors from a shipwreck. Made with C and the Allegro multimedia library (graphics and input). \href{http://mikail.io/castaway/}{[link]}

    {\myprojectface Augmented reality 3D game with ActionScript 3} \hfill 2010 (6 months) \\
    {\mypositionface Solo Programmer} \\
    Augmented reality 3D puzzle game where the player has to make a ball reach the objective going through obstacles in the virtual inside of a cube. Made with ActionScript 3 and the following libraries: JigLib for 3D physics, Papervision3D for 3D graphics, FLARManager for augmented reality. \href{http://mikail.io/ar_game/}{[link]}
\else
\fi

{\myprojectface First Person Walking Simulator with C++ and OpenGL} \hfill 2012 (5 months) \\
{\mypositionface Solo Programmer} \\
Interactive simulation with basic physics enabling the user to walk around the buildings of the USP's Institute of Mathematics and Statistics (outside only) with optional simulation for rain and fog. Made solely with C++ and OpenGL. \ifgames \href{http://mikail.io/walker/}{[link]} \fi

%-----------------------------------------------------------------------------------------------------------------------------------------------------------------------------

\section{PERSONAL \hspace{\divspace} }
\section{PROJECTS \hspace{\divspace} }
\vskip-2.5ex  % mágica da gambi que faz o texto ser renderizado uma linha pra cima

{\myprojectface Multiplatform game with C++} \hfill 2012 — present (sparse) \\
{\mypositionface Solo Programmer} \\
2D platform game with proprietary physics and lightning effects engines. Multiplatform project that utilizes OpenGL, GLSL and the SFML (C++) library for input and graphics management.
Currently has a version in development stage for Mac OS X. \ifgames \href{http://mikail.io/platformer/}{[link]} \fi

{\myprojectface iOS game with Swift} \hfill 2015 — present (sparse) \\
{\mypositionface Solo Programmer} \\
2D platform {\it infinite runner} game using the SpriteKit framework from Apple for graphics and physics, with proprietary engine for procedural content generation. \ifgames \href{http://mikail.io/runner/}{[link]} \fi

%-----------------------------------------------------------------------------------------------------------------------------------------------------------------------------

\section{EXPERIENCE \hspace{\divspace} }

{\mypositionface Mobile Developer} \hfill 2013 - 2014 \\
{\myorgface VTX Brasil}

\begin{itemize}[itemsep=\myitemsep,leftmargin=\myitemback]
\item[\sbt] {\it iOS application for IM and video calls} \hfill (6 months)\\
Development of the user interface and integration with the Liblinphone library for SIP ({\it Session Initiation Protocol}) based communication.
\item[\sbt] {\it Android application for multiple-device split video playback application} \hfill (3 months)\\
Development of the user interface and integration with the AllJoyn framework for communication between the devices and management of the playback synchronization in each device using NTP ({\it Network Time Protocol}).
\end{itemize}

{\mypositionface Lead iOS Developer} \hfill 2015\\
{\myorgface VTX Brasil}

\begin{itemize}[itemsep=\myitemsep,leftmargin=\myitemback]
\item[\sbt] {\it Product conception and direction of development}
\item[\sbt] {\it Definition and establishment of iOS development standards, as well as a reusable code base, to ensure quality and agility of the team's code production}

\ifgames
    \vfill
\else
\fi

\item[\sbt] {\it iOS application for engagement in retail through innovative technologies} \hfill (20 months)\\
Application conception, architecting and development. Application conception in conjunction with the client and the design team. App development and integration with our proprietary framework for information extension of printed media, as well as simple beacon management for integration with web application enabling salespeople to access basic information of stores' visitors.
\item[\sbt] {\it iOS framework for digital extension of information in printed media} \hfill (4 months)\\
Conception, architecting and development of the extension framework, which in turn was based on the augmented reality framework Metaio. iOS front of an ecosystem of tools that aimed to integrate with the printed media production process of business clients to digitally extend it.

\ifgames
\else
    \vfill
\fi

\item[\sbt] {\it iOS application for advertisement of retail deals} \hfill (4 months)\\
Application architecting, development and management of development. App core implementation and management of implementation of the integration with our proprietary server for accessing retail information.
\end{itemize}

{\mypositionface iOS Developer} \hfill 2016\\
{\myorgface QuintoAndar}

\begin{itemize}[itemsep=\myitemsep,leftmargin=\myitemback]
\item[\sbt] {\it Product architecting}
\item[\sbt] {\it Creation of recurrent iOS discussions for the definition and establishment of standards as well as new practices in development (such as MVVM)}

\item[\sbt] {\it iOS application for searching and visiting apartments for rent} \hfill (5 months)\\
Application architecting and development. {\it This app received the Best of 2016 prize from Apple.} 
\item[\sbt] {\it iOS application for managing user's apartments for rent} \hfill (4 months)\\
Application development and refactoring. Development in pursue of eliminating current technical debts as well as refactoring current codebase to ensure the best and more modern practices were being used.

\ifgames
\else
\fi
\end{itemize}

{\mypositionface Mobile Tech Lead} \hfill 2017\\
{\myorgface Itaú}

\begin{itemize}[itemsep=\myitemsep,leftmargin=\myitemback]
\item[\sbt] {\it Analysis and engineering of the best solutions within Itaú's technical ecosystem}
\item[\sbt] {\it Part of iOS discussions for best practices and part of the team responsible for guaranteeing the implementation of coding standards by other feature squads}
\item[\sbt] {\it iOS application for serving business users} \hfill (current)\\
Management of iOS and Android development as well as necessary integration with Itaú's backend.

\ifgames
\else
\fi
\end{itemize}

%-----------------------------------------------------------------------------------------------------------------------------------------------------------------------------

\section{SKILLS \hspace{\divspace} }

- {\sl Languages and syntaxes}: \\
\setlength\tabcolsep{2pt}
\begin{tabular}{r l}
\hspace*{1.5em}{\sl Fluent}:& Swift, Objective-C, C, regex \\
\hspace*{1.5em}{\sl More Familiar}:& Python, C++, Java, Scheme, Erlang, ActionScript 3, \LaTeX, GLSL \\
\hspace*{1.5em}{\sl Less Familiar}:& CUDA, Assembly, Prolog, Smalltalk, Pascal, Bash \\
\end{tabular}
\\
- {\sl Frameworks, libraries and technologies:} \\
\hspace*{1.5em}iBeacon, SFML, Allegro, NumPy, JigLib, Papervision3D, FLARManager, Vuforia, \\
\hspace*{1.5em}OpenCV, Augmented Reality
\\
- {\sl Softwares and tools:} \\
\hspace*{1.5em}Git (advanced), Docker (basic), gnuplot, Xcode, Eclipse, ADT, FlashDevelop, Sublime Text, TextMate, vim
\\
- {\sl Architectures:} \\
\hspace*{1.5em}MVC, MVVM
\\
- Intermediate knowledge of OpenGL and shaders \\
- Basic knowledge of cryptography and Artificial Intelligence \\
- Development in Ubuntu and Debian systems (4 years) and Mac OS X (5 years) \\
- Intermediate-advanced image editing in Adobe Photoshop \\
- Fluent english, basic japanese

%-----------------------------------------------------------------------------------------------------------------------------------------------------------------------------

\section{AWARDS \hspace{\divspace} }

- Bronze medal in the {\sl Informatics Brazilian Olympics} (2007) \\
- Silver medal in the {\sl Astronomy Brazilian Olympics} (2006) \\
- Silver medal in the {\sl Astronomy Brazilian Olympics} (2007) \\
- Honorable Mention in the {\sl Physics Brazilian Olympics} (2007) \\

%----------------------------------------------------------------------------------------

\end{resume}
\end{document}