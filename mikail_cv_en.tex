%%%%%%%%%%%%%%%%%%%%%%%%%%%%%%%%%%%%%%%%%
% Medium Length Graduate Curriculum Vitae
% LaTeX Template
% Version 1.1 (9/12/12)
%
% This template has been downloaded from:
% http://www.LaTeXTemplates.com
%
% Original author:
% Rensselaer Polytechnic Institute (http://www.rpi.edu/dept/arc/training/latex/resumes/)
%
% Important note:
% This template requires the res.cls file to be in the same directory as the
% .tex file. The res.cls file provides the resume style used for structuring the
% document.
%
%%%%%%%%%%%%%%%%%%%%%%%%%%%%%%%%%%%%%%%%%

%----------------------------------------------------------------------------------------
%	PACKAGES AND OTHER DOCUMENT CONFIGURATIONS
%----------------------------------------------------------------------------------------

\documentclass[]{res} % Use the res.cls style, the font size can be changed to 11pt or 12pt here
% \documentclass[a4paper]{article}

\usepackage{helvet} % Default font is the helvetica postscript font
% \usepackage{newcent} % To change the default font to the new century schoolbook postscript font uncomment this line and comment the one above
% \usepackage{baskervald}

\usepackage[brazil]{babel}
\usepackage[utf8]{inputenc}
% \usepackage[tmargin=2.2cm, bmargin=2.2cm, lmargin=2.2cm, rmargin=2.2cm]{geometry}
\usepackage[pdftex]{hyperref}
\hypersetup{
    colorlinks,
    citecolor=black,
    filecolor=black,
    linkcolor=blue,
    urlcolor=blue
}

\usepackage{calc}

\usepackage{enumitem}

% A4 tem 8.267in x 11.692in

% resumewidth default é 6.5

% nt = nr - vr + vt

\def \mymargin{0.4in}
\def \mymarginright{0.85in}
\def \mysectionwidth{70pt}

\newsectionwidth{\mysectionwidth}

\def \newresumewidth{8.267in - \mymargin - \mymarginright}
\def \newtextwidth{\newresumewidth - \mysectionwidth}

\usepackage[margin=\mymargin, top=0.7in, a4paper]{geometry}

\setlength{\textwidth}{\newtextwidth}
% \setlength{\textwidth}{5.1in}
\setlength{\resumewidth}{\newresumewidth}

\newcommand{\sbt}{\,\begin{picture}(-1,1)(-2,-3)\circle*{2}\end{picture}\ }

\def \divspace{6pt}
\def \myitemback{0.55cm}
\def \myitemsep{0pt}

% \def \mypositionface{\sc}
\def \mypositionface{\sl}
% \def \mypositionface{\it}
% \def \mypositionface{\uppercase}
% \def \mypositionface{\lfseries}
% \def \mypositionface{\mdseries}

\def \myprojectface{\it}

\def \myorgface{\sc}
% \def \myorgface{\sl}
% \def \myorgface{\it}
% \def \myorgface{\uppercase}
% \def \myorgface{\lfseries}
% \def \myorgface{\mdseries}

\begin{document}

%----------------------------------------------------------------------------------------
%	NAME AND ADDRESS SECTION
%----------------------------------------------------------------------------------------

\moveleft.5\hoffset\centerline{\huge\uppercase{Mikail Campos Freitas}} % Your name at the top

\moveleft.5\hoffset\centerline{\rule{2.5in}{.2pt}}\smallskip % Your address

\moveleft.5\hoffset\centerline{Addr. Alameda Itupiranga, 309, Saúde} % Your address
\moveleft.5\hoffset\centerline{Brazil, São Paulo, SP - 04294-090}
\moveleft.5\hoffset\centerline{Home. +55 (11) 2352-2157 / Mobile. +55 (11) 97288-9984}
\moveleft.5\hoffset\centerline{contact@mikail.io / Website. \href{http://mikail.io}{http://mikail.io}}
\vskip-3.5ex

%-----------------------------------------------------------------------------------------------------------------------------------------------------------------------------

\begin{resume}
\section{OBJECTIVES \hspace{\divspace} }

% Aprender novas tecnologias e expandir conhecimentos e habilidades, através da busca e superação de desafios e da troca de aprendizado com outros profissionais.
To learn new technologies and to expand my knowledge and skills through the pursuit and overcoming of challenges. To exchange experience with other professionals and becoma a better one myself.

%-----------------------------------------------------------------------------------------------------------------------------------------------------------------------------
\section{EDUCATION \hspace{\divspace} }

% {\myprojectface Bacharelado em Ciência da Computação} \hfill 2008 — 2012 \\
% {\myorgface Instituto de Matemática e Estatística} da {\myorgface Universidade de São Paulo} \\
% Trabalho de Conclusão de Curso: {\sl Resolvendo o problema PSAT com auxílio da ferramenta de software livre MiniSat} \href{http://www.ime.usp.br/~cef/mac499-12/monografias/rec/mikail/index.html}{[link]}

{\myprojectface Bachelor of Computer Science} \hfill 2008 — 2012 \\
{\myorgface University of São Paulo}'s {\myorgface Institute of Mathematics and Statistics} (IME - USP)\\
Course Completion Assignment: {\sl Solving the PSAT problem with assistance of the free software MiniSat} \href{http://www.ime.usp.br/~cef/mac499-12/monografias/rec/mikail/index.html}{[link]}

%-----------------------------------------------------------------------------------------------------------------------------------------------------------------------------
\section{COLLEGE \hspace{\divspace} }
\section{PROJECTS \hspace{\divspace} }
\vskip-2.5ex  % mágica da gambi que faz o texto ser renderizado uma linha pra cima

{\myprojectface Simple 2D game with C} \hfill 2010 (4 months) \\
{\mypositionface Lead Programmer} \\
2D game with simple physics where two players compete with each other for points for saving survivors from a shipwreck. Made with C and the Allegro multimedia library (graphics and input). \href{http://mikail.io/castaway/}{[link]}

{\myprojectface Augmented reality 3D game with ActionScript 3} \hfill 2010 (6 months) \\
{\mypositionface Solo Programmer} \\
Augmented reality 3D puzzle game where the player has to make a ball reach the final destination through obstacles in the virtual inside of a cube. Made with ActionScript 3 and the following libraries: JigLib for 3D physics, Papervision3D for 3D graphics, FLARManager for augmented reality. \href{http://mikail.io/ar_game/}{[link]}

% {\myprojectface Desenvolvimento de aplicação interativa 3D simples em C++ com OpenGL} \hfill 2012 (5 meses) \\
% {\mypositionface Programador Solo} \\
% Simulação interativa de física simples onde é possível andar pelos prédios do Instituto de Matemática e Estatística da USP (ambiente externo). Feita em C++ com OpenGL.

{\myprojectface First Person Walking Simulator with C++ and OpenGL} \hfill 2012 (5 months) \\
{\mypositionface Solo Programmer} \\
Interactive simulator with basic physics enabling the user to walk around the buildings of the USP's Institute of Mathematics and Statistics (outside only). Made solely with C++ and OpenGL. \href{http://mikail.io/walker/}{[link]}

%-----------------------------------------------------------------------------------------------------------------------------------------------------------------------------
\section{PERSONAL \hspace{\divspace} }
\section{PROJECTS \hspace{\divspace} }
\vskip-2.5ex  % mágica da gambi que faz o texto ser renderizado uma linha pra cima

% {\myprojectface Jogo multiplataforma em C++} \hfill 2012 — presente \\
% {\mypositionface Programador Solo} \\
% Jogo 2D com {\it engine} próprio para física e efeitos de luz e sombra. Projeto multiplataforma usando a biblioteca multimídia SFML (C++), OpenGL e GLSL. Atualmente possui versão para Mac OS X e Windows.

{\myprojectface Multiplatform game with C++} \hfill 2012 — present (very sparse) \\
{\mypositionface Solo Programmer} \\
2D platform game with it's own physics and lightning effects egines. Multiplatform project that utilizes OpenGL, GLSL and the SFML (C++) library for input and graphics management.
Currently has a version in {\it very} early development for Mac OS X. Windows version in progress.

%-----------------------------------------------------------------------------------------------------------------------------------------------------------------------------
\section{EXPERIENCE \hspace{\divspace} }

% {\mypositionface Desenvolvedor Pleno} \hfill 2013 — 2014 \\
% {\myorgface VTX Brasil}

% \begin{itemize}[itemsep=\myitemsep,leftmargin=\myitemback]
% \item[\sbt] Participação na concepção e direção do desenvolvimento de produtos.
% \item[\sbt] Desenvolvimento de {\it frontend} e {\it backend} em aplicativo iOS de mensagens instantâneas e vídeo-chamadas. (6 meses)
% \item[\sbt] Desenvolvimento de {\it frontend} e {\it backend} em aplicativo Android de multivisualização de vídeos.\\ (3 meses)
% \item[\sbt] Desenvolvimento de {\it frontend} e {\it backend} em aplicativo iOS de engajamento em varejo através de tecnologias de ponta. (10 meses)
% \end{itemize}

% {\mypositionface Líder Técnico de P\&D} \hfill 2014 — presente \\
% {\myorgface VTX Brasil}

% \begin{itemize}[itemsep=\myitemsep,leftmargin=\myitemback]
% \item[\sbt] Planejamento e desenvolvimento de sistemas inteligentes de aprendizado computacional com ênfase em processamento de imagens. (2 meses)
% \end{itemize}

{\mypositionface Programmer} \hfill 2013 — 2014 \\
{\myorgface VTX Brasil}

\begin{itemize}[itemsep=\myitemsep,leftmargin=\myitemback]
\item[\sbt] Development of IM and video call application for iOS. (6 months)
\item[\sbt] Development of multiple-device split video playback application for Android. (3 months)
\end{itemize}

{\mypositionface Lead Programmer} \hfill 2014 — present \\
{\myorgface VTX Brasil}

\begin{itemize}[itemsep=\myitemsep,leftmargin=\myitemback]
\item[\sbt] Product conception and direction of development.
\item[\sbt] Head of development of iOS application for engagement in retail through innovative technologies. (14 months)
\end{itemize}

%-----------------------------------------------------------------------------------------------------------------------------------------------------------------------------
\section{SKILLS \hspace{\divspace} }

% - {\sl Linguagens}: \\
% \setlength\tabcolsep{2pt}
% \begin{tabular}{l l}
% \hspace*{1.5em}{\it Fluente}:& Objective-C, C, regex \\
% \hspace*{1.5em}{\it Muito Familiar}:& Python, C++, Java, Scheme, Erlang, ActionScript 3, \LaTeX, GLSL \\
% \hspace*{1.5em}{\it Pouco Familiar}:& CUDA, Assembly, Prolog, Swift, Smalltalk, Pascal, Bash \\
% \end{tabular}
% \\
% - {\sl Frameworks, bibliotecas e tecnologias:} \\
% \hspace*{1.5em}iBeacon, SFML, Allegro, NumPy, JigLib, Papervision3D, FLARManager, AllJoyn, Vuforia, \\
% \hspace*{1.5em}OpenCV, Liblinphone, Realidade Aumentada
% \\
% - {\sl Softwares e ferramentas:} \\
% \hspace*{1.5em}Git (avançado), gnuplot, Xcode, Eclipse, ADT, FlashDevelop, Sublime Text, TextMate, vim
% \\
% - Conhecimento intermediário de OpenGL e {\it shaders} \\
% - Desenvolvimento em Ubuntu e Debian (7 anos), e Mac OS X (5 anos) \\
% - Conhecimento intermediário de criptografia \\
% - Edição de imagem intermediário-avançada no Adobe Photoshop \\
% - Inglês fluente, japonês básico-intermediário, espanhol básico


- {\sl Languages}: \\
\setlength\tabcolsep{2pt}
\begin{tabular}{l l}
\hspace*{1.5em}{\it Fluent}:& Objective-C, C, regex \\
\hspace*{1.5em}{\it More Familiar}:& Python, C++, Java, Scheme, Erlang, ActionScript 3, \LaTeX, GLSL \\
\hspace*{1.5em}{\it Less Familiar}:& CUDA, Assembly, Prolog, Swift, Smalltalk, Pascal, Bash \\
\end{tabular}
\\
- {\sl Frameworks, libraries and technologies:} \\
\hspace*{1.5em}iBeacon, SFML, Allegro, NumPy, JigLib, Papervision3D, FLARManager, AllJoyn, Vuforia, \\
\hspace*{1.5em}OpenCV, Liblinphone, Augmented Reality
\\
- {\sl Softwares e tools:} \\
\hspace*{1.5em}Git (advanced), gnuplot, Xcode, Eclipse, ADT, FlashDevelop, Sublime Text, TextMate, vim
\\
- Intermediate knowledge of OpenGL and shaders \\
- Development in Ubuntu and Debian systems (7 years) and Mac OS X (5 years) \\
- Intermediate knowledge of cryptography \\
- Intermediate-advanced image editing in Adobe Photoshop \\
- Fluent english, basic-intermediate japanese, basic spanish\\\\

%-----------------------------------------------------------------------------------------------------------------------------------------------------------------------------
% \section{PRÊMIOS E \hspace{\divspace} }
% \section{CONQUISTAS \hspace{\divspace} }
% \vskip-2.5ex  % mágica da gambi que faz o texto ser renderizado um linha pra cima

% - Medalha de Bronze na {\sl Olimpíada Brasileira de Informática} (2007) \\
% - Medalha de Prata na {\sl Olimpíada Brasileira de Astronomia} (2006) \\
% - Medalha de Prata na {\sl Olimpíada Brasileira de Astronomia} (2007) \\
% - Menção Honrosa na {\sl Olimpíada Brasileira de Física} (2007) \\
% - Chamado até a última fase do recrutamento da {\sl Microsoft Corporation} no IME - USP (2012)

\section{AWARDS \hspace{\divspace} }

- Bronze medal in the {\sl Informatics Brazilian Olympics} (2007) \\
- Silver medal in the {\sl Astronomy Brazilian Olympics} (2006) \\
- Silver medal in the {\sl Astronomy Brazilian Olympics} (2007) \\
- Honorable Mention in the {\sl Physics Brazilian Olympics} (2007) \\
- Progressed to the last stage of recruitment by the Microsoft Corporation in IME - USP (2012)

%----------------------------------------------------------------------------------------

\end{resume}
\end{document}